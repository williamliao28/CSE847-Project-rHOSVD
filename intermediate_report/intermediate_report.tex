\documentclass[12pt,a4paper]{article}
\usepackage{natbib}
\usepackage{bibentry}
\usepackage{amsmath}
\usepackage{amssymb}
\usepackage{amsthm}
\usepackage{amsfonts}
\usepackage{graphicx}
\usepackage{subfigure}
\usepackage{float}
\usepackage{algorithm}
\usepackage{algorithmic}
% Packages for table
\usepackage{multirow}
\usepackage{longtable}
\usepackage{makecell}
\usepackage{array}
\usepackage{booktabs}
\usepackage{hyperref}
\usepackage{comment}
\usepackage{ulem}
\usepackage{etoolbox}
\usepackage{tikz}
\usepackage{geometry}

%\geometry{a4paper,left=1cm,right=1cm,top=1cm,bottom=1.5cm}

\usepackage{ulem}
\usepackage{color}
\newcommand{\rd}{\textcolor{red}}
\newcommand{\gr}{\textcolor{Green}}
\newcommand{\og}{\textcolor{Orange}}
\newcommand{\ce}{\textcolor{Cerulean}}
\newcommand{\tb}{\textcolor{blue}}

\hypersetup{
colorlinks=true,
allcolors=.
}

\title{\textbf{\large{CSE 847 (Spring 2022): Machine Learning\\ Intermediate Report}}}
\author{
    Wei-Chien Liao (\href{mailto:liaowei2@msu.edu}{liaowei2@msu.edu})
    \and
    Shihab Shahriar Khan (\href{mailto:khanmd@msu.edu}{khanmd@msu.edu})
}

\date{}

\begin{document}
\maketitle
\begin{enumerate}
    \item \textbf{Project Title:}\\
    A Comparison of Various Randomized Higher Order Singular Value Decomposition (HOSVD) Algorithms
    \item \textbf{Team Members:} Wei-Chien Liao, Shihab Shahriar Khan
    \item \textbf{Introduction and Problem Description:}\\
    Many applications in data sciences require processing high-order tensor data. To deal
    with large tensor data, dimensionality reduction techniques play an important role among many other types of algorithms. However,
    performing dimension reduction operations like Tucker decomposition and High Order Singular Value Decomposition (HOSVD)
    with deterministic algorithms are not efficient for handling large tensor data. This inefficiency can be
    handled by randomized algorithms. This type of algorithms accelerate classical decompostions by reducing computational complexity
    of deterministic methods and communications among different level of memory hierarchy. This project aims to study, implement and compare
    many variants of randomized algorithms, and test them with different datasets from applications such as handwritten digit classification, computer vision or signal processing 
    to evaluate their performances.
    \item \textbf{Description of the data used in the project:}
    \begin{enumerate}
        \item Synthetic 3rd order tensor, optionally with noise added.
        \item COIL-100 DATA SET \cite{nene1996columbia}: contains 7200 color images (100 objects under 72 different rotations). 
        \item NEIL 2 \cite{carlson2010toward}: For experiments on sparse tensor. It's a 3rd order sparse tensor of shape $12092 \times 9184 \times 28818$.
    \end{enumerate}
    
    
    \item \textbf{What We Have Done So Far:}
    \begin{enumerate}
        \item Study the theory and mathematics of two prominent deterministic tensor decomposition algorithms called Tucker decomposition and High Order
        Singular Value Decomposition (HOSVD).
        \item Study the papers \cite{9350569} and \cite{Kolda2009} to understand randomized tensor decomposition algorithms, and how they perform compared to their deterministic counterparts along several metrics of interest.
        \item Explore the Tensor Toolbox for MATLAB \cite{Brett2021}.
        \begin{enumerate}
            \item Test the \texttt{HOSVD} functionality in the Toolbox.
            \item Implement STHOSVD and HOOI algorithm.
        \end{enumerate} 
        \item Exploration of ``Tensorly" \cite{tensorly} library in Python.
        \begin{enumerate}
            \item Implementation of tensor decomposition and reconstruction using lower level functions to gain a better idea of how the library and the algorithms work.
            \item Using order-3 tensors of image and synthetic data, implemented a preliminary experiment to test the reconstruction performance of 3 algorithms: CP, Tucker and RandomizedCP. The algorithms were evaluated based on Relative Error and Running Time, following the procedure of \cite{9350569}. The qualitative performance on image data was also considered. 
        \end{enumerate}
        
    \end{enumerate}
    
    \item \textbf{What Remains to be Done:}
    \begin{enumerate}
        \item Implement the randomized algorithms in \cite{9350569}.
        \item Analyze the algorithms using the datasets listed in
        the ``Description of the data'' section.
        \item Propose and evaluate potential ways further speedup can be achieved.
    \end{enumerate}
    
\end{enumerate}

\bibliographystyle{unsrt} % We choose the "plain" reference style
\bibliography{msu-cse847-project.bib}

\end{document}