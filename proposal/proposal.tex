\documentclass[12pt,a4paper]{article}
\usepackage{natbib}
\usepackage{bibentry}
\usepackage{amsmath}
\usepackage{amssymb}
\usepackage{amsthm}
\usepackage{amsfonts}
\usepackage{graphicx}
\usepackage{subfigure}
\usepackage{float}
\usepackage{algorithm}
\usepackage{algorithmic}
% Packages for table
\usepackage{multirow}
\usepackage{longtable}
\usepackage{makecell}
\usepackage{array}
\usepackage{booktabs}
\usepackage{hyperref}
\usepackage{comment}
\usepackage{ulem}
\usepackage{etoolbox}
\usepackage{tikz}
\usepackage{geometry}

%\geometry{a4paper,left=1cm,right=1cm,top=1cm,bottom=1.5cm}

\usepackage{ulem}
\usepackage{color}
\newcommand{\rd}{\textcolor{red}}
\newcommand{\gr}{\textcolor{Green}}
\newcommand{\og}{\textcolor{Orange}}
\newcommand{\ce}{\textcolor{Cerulean}}
\newcommand{\tb}{\textcolor{blue}}

\hypersetup{
colorlinks=true,
allcolors=.
}

\title{\textbf{\large{CSE 847 (Spring 2022): Machine Learning --- Project Proposal}}}
\author{Wei-Chien Liao (\href{mailto:liaowei2@msu.edu}{liaowei2@msu.edu})}
\date{}

\begin{document}
\bibliographystyle{unsrt}
\nobibliography{msu-cse847-project}
\maketitle
\begin{enumerate}
    \item \textbf{Project Title:}\\
    A Comparison of Various Randomized Higher Order Singular Value Decomposition (HOSVD) Algorithms
    \item \textbf{Team Members:} Wei-Chien Liao, Shihab Shahriar Khan
    \item \textbf{Description of the Problem:}\\
    Many applications in data sciences require processing high-order tensor data. To deal
    with large tensor data, dimensionality reduction techniques play an important role among many other types of algorithms. However,
    performing dimension reduction operations like Tucker decomposition and High Order Singular Value Decomposition (HOSVD)
    with deterministic algorithms are not efficient for handling large tensor data. This inefficiency can be
    handled by randomized algorithms. This type of algorithms accelerate classical decompostions by reducing computational complexity
    of deterministic methods and communications among different level of memory hierarchy. This project aims to study, implement and compare
    many variants of randomized algorithms, and test them with different datasets from applications such as handwritten digit classification, computer vision or signal processing 
    to evaluate their performances.
    \item \textbf{Preliminary Plan (Milestones):}
    \begin{enumerate}
        \item Test Tensor Toolbox for MATLAB
        \item Study the paper \cite{9350569} in the paper list.
        \item Implement, analyze and compare the algorithms in \cite{9350569}
    \end{enumerate}
    \item \textbf{Paper List:}
    \begin{enumerate}
        \item[\cite{9350569}] \bibentry{9350569}
        \item[\cite{ma2021fast}] \bibentry{ma2021fast}
        \item[\cite{Brett2021}] \bibentry{Brett2021}
        \item[\cite{Minster2020}] \bibentry{Minster2020}   
        \item[\cite{Kolda2009}] \bibentry{Kolda2009}
    \end{enumerate}
\end{enumerate}
\end{document}